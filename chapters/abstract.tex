%!TEX root = ../thesis.tex

\abstract
%%%%%%%%%%%%%%%%%%%%%%%%%%%%%%%%%%%%%%%%%%%%%%%%%%%%%%
%%
%%  					中文摘要
%%
%%%%%%%%%%%%%%%%%%%%%%%%%%%%%%%%%%%%%%%%%%%%%%%%%%%%%%
{
深对流云是大气质量垂直输送的主要载体,它能够使含有各种污染气体在相对较短的时间内
由边界层源区输送到对流层上层甚至平流层低层,影响区域和全球大气环境和气候。
此外闪电氮氧化物(LNO$_x$)为对流层上层氮氧化物(NO$_x$)的主要来源,对臭氧(O$_3$)化学产生直接和间接的影响,
目前LNO$_x$的产率、分布及其影响仍具有很高的不确定性,
基于探空和飞机观测等传统方法的对流观测试验难度较大,而卫星遥感技术的发展为定量排放提供了新方法。
本研究围绕这一特点,通过卫星资料分析、数值模拟和外场观测试验,
确定了不同污染地区的LNO$_x$产率和产量,
对比分析了观测和模拟的二氧化氮(NO$_2$)及O$_3$垂直分布,
揭示了动力输送和化学反应对对流层上层NO$_x$和O$_3$浓度的相对重要性。

首先,基于对流层观测仪(TROPOMI)和臭氧监测仪(OMI)的NO$_2$观测,
针对清洁地区(北极)和污染地区(北美大陆和中国东南部)的旺盛对流,
分别定义了各自用于计算闪电二氧化氮(LNO$_2$)柱浓度的大气质量因子,
结果显示在小时尺度上LNO$_2$柱浓度与重污染地区的NO$_2$柱浓度相当。
北极海洋上的LNO$_2$产率是北极陆地性闪电的6倍,基于此获得了北极夏季LNO$_x$的平均排放量,而北极地区与污染地区的LNO$_2$产率相当。
除旺盛对流之外,消散期的对流系统仍存在NO$_2$高值像素数据,进一步得到北极地区的LNO$_2$寿命和中国东南部的LNO$_2$产率。

其次,利用TROPOMI针对不同高度对流云的NO$_2$和O$_3$观测,
获得了中低纬度地区对流云内NO$_2$和O$_3$的垂直分布。
通过对比TROPOMI观测数据和全球化学模式结果,验证了LNO$_2$在对流上层的主导地位,
揭示了全球模式对于LNO$_2$和污染物垂直输送模拟的缺陷,
指出了该廓线信息可用于评估化学传输模型的结果,并指导对流参数化方案的开发。

最后,综合TROPOMI观测、WRF-Chem模拟和O$_3$探空试验,
定量分析了中国东南部典型对流系统中动力输送、化学反应和化学反应速率的时空变化。
结果表明,虽然在对流旺盛期间动力输送项主导O$_3$的浓度变化,但在整个生命期中化学反应更为重要。
其中LNO$_x$使得对流层上层NO$_2$浓度和累积净化学反应速率增大,但动力输送和化学反应的综合作用依然降低了对流层上层O$_3$浓度。
}
%%%%%%%%%%%%%%%%%%%%%%%%%%%%%%%%%%%%%%%%%%%%%%%%%%%%%%
%%
%%  					英文摘要
%%
%%%%%%%%%%%%%%%%%%%%%%%%%%%%%%%%%%%%%%%%%%%%%%%%%%%%%%
{

}
