%!TEX root = ../thesis.tex

\abstract
%%%%%%%%%%%%%%%%%%%%%%%%%%%%%%%%%%%%%%%%%%%%%%%%%%%%%%
%%
%%  					中文摘要
%%
%%%%%%%%%%%%%%%%%%%%%%%%%%%%%%%%%%%%%%%%%%%%%%%%%%%%%%
{
深对流云是大气垂直输送的主要载体,它能将污染气体在较短时间内,
由边界层源区输送至对流层上层甚至平流层低层,从而影响区域和全球的大气环境和气候。
此外闪电氮氧化物(LNO$_{\ch{x}}$)是对流层上层氮氧化物(NO$_{\ch{x}}$)的主要来源,
对臭氧(O$_{\ch{3}}$)化学有着直接和间接的影响。
目前LNO$_{\ch{x}}$的产率及其化学影响仍有很大的不确定性,
由于探空和飞机观测等传统试验的难度较大,
所以卫星遥感技术的发展为定量计算其排放提供了新方法。
本研究围绕这一特点,通过卫星资料反演、数值模拟和臭氧探空试验,
确定了污染和清洁地区的闪电二氧化氮(LNO$_{\ch{2}}$)和LNO$_{\ch{x}}$产率,
对比分析了观测和模拟的NO$_{\ch{2}}$及O$_{\ch{3}}$垂直分布,
揭示了深对流的动力输送和化学反应在对流层上层NO$_{\ch{2}}$和O$_{\ch{3}}$浓度变化过程中各自的贡献。

首先,利用星载对流层观测仪(TROPOMI)和臭氧监测仪(OMI)的遥感观测资料,
开发了同时适用于污染和清洁地区的LNO$_{\ch{2}}$和LNO$_{\ch{x}}$柱浓度反演算法。
通过结合该柱浓度产品和地基闪电观测资料,
得到了北美大陆地区旺盛对流的LNO$_{\ch{2}}$和LNO$_{\ch{x}}$产率,
分别为32$\pm$15 mol NO$_{\ch{2}}$每闪电和90$\pm$50 mol NO$_{\ch{x}}$每闪电。
将该结果与前人在北美地区的星载观测研究进行对比验证,揭示了本研究中反演算法的普适性。
除普遍研究的旺盛对流之外,该反演算法也适用于对流的初始和消散阶段,
弥补了TROPOMI在旺盛对流处因饱和效应而缺失数据的缺点,
结果显示中国东南部消散对流的LNO$_{\ch{x}}$产率为60$\pm$33 mol NO$_{\ch{x}}$每闪电。
清洁的北极陆地地区LNO$_{\ch{2}}$产率为2.0 mol每闪击,与中纬度污染地区的LNO$_{\ch{2}}$产率相当,
而北极海洋地区的LNO$_{\ch{2}}$ 产率是北极陆地性闪电的6倍,
总体来看北极夏季的LNO$_{\ch{x}}$排放量是该地区人为NO$_{\ch{x}}$排放的5\%。

其次,利用TROPOMI针对不同高度对流云的观测,
揭示了中低纬度对流云内的“C”型NO$_{\ch{2}}$垂直分布廓线,
具体而言,陆地地区对流层顶--330 hPa间的NO$_{\ch{2}}$浓度为450--570hPa间的$\sim$2倍,570hPa以下的NO$_{\ch{2}}$浓度逐渐上升,
即LNO$_{\ch{2}}$在对流层上层占主导,而污染 NO$_{\ch{2}}$ 在下对流层占主导。
通过对比TROPOMI观测数据和全球模式资料,
发现有云条件下中纬度地区对流层上层O$_{\ch{3}}$浓度比晴空条件下低26\%,
低纬度海洋地区O$_{\ch{3}}$浓度较晴空低17\%,
而非洲中部和印度北部的O$_{\ch{3}}$平均浓度较晴空增大20\%。
此外,全球模式对于LNO$_{\ch{2}}$和人为污染物垂直输送的低估,
导致对流层上层NO$_{\ch{2}}$模拟浓度偏低10--50\%。
因此TROPOMI观测获得的廓线信息可用于评估模式并指导参数化方案的开发。

最后,综合TROPOMI观测、WRF-Chem模拟和O$_{\ch{3}}$探空试验数据,
定量分析了中国东南部典型对流系统中动力输送、化学反应和化学反应速率的时空变化。
结果表明,
虽然在对流旺盛期间动力输送项主导O$_{\ch{3}}$的浓度变化,
但在整个生命期中化学反应可达动力输送的5--10倍。
此外敏感性试验表明,LNO$_{\ch{x}}$可使得对流层上层O$_{\ch{3}}$化学累积生成速率降低 4\%,累积消耗速率增加23\%,
最终导致该层O$_{\ch{3}}$的平均 浓度降低了25\%。
若将对流分为核心区和层云区,则核心区的动力输送作用为层云区的$\sim$2倍,层云区的O$_{\ch{3}}$变化受核心区的输送贡献所控制。而LNO$_{\ch{x}}$使得核心区的O$_{\ch{3}}$化学产量增加125\%,但净产量下降21\%。
% 若将LNO$_{\ch{2}}$考虑进TROPOMI NO$_{\ch{2}}$反演所使用的先验廓线,
% 则对流层空气质量因子在新生闪电区降低23\%,
% 而在出流区和老化区增大60\%。
}
%%%%%%%%%%%%%%%%%%%%%%%%%%%%%%%%%%%%%%%%%%%%%%%%%%%%%%
%%
%%  					英文摘要
%%
%%%%%%%%%%%%%%%%%%%%%%%%%%%%%%%%%%%%%%%%%%%%%%%%%%%%%%
{
Deep convection clouds play a crucial role in transporting mass and pollutants in the atmosphere, impacting regional and global environment.
Lightning nitrogen oxides (LNO$_{\ch{x}}$) are the main source of nitrogen oxides (NO$_{\ch{x}}$) in the upper troposphere
and affect ozone (O$_{\ch{3}}$) chemistry, yet their production, distribution, and impact remain uncertain.
Conventional observation methods (e.g. sounding and aircraft observations) are challenging for convection studies,
but satellite remote sensing provides new opportunities for quantifying emissions.
This study uses satellite data analysis, numerical simulation, and field experiments to determine LNO$_{\ch{x}}$ production efficiencies and emissions in clean and polluted regions,
and analyze the observed and simulated nitrogen dioxide (NO$_{\ch{2}}$) and O$_{\ch{3}}$ vertical distributions.
It sheds light on the relative impact of transportation and chemical reactions on NO$_{\ch{2}}$ and O$_{\ch{3}}$ concentration in the upper troposphere.

Firstly, we analyze NO$_{\ch{2}}$ observations from the TROPOspheric Monitoring Instrument (TROPOMI) and
Ozone Monitoring Instrument (OMI) in various regions, including clean areas such as the Arctic, and polluted regions such as North America and southeastern China, during periods of active convection.
We determine air mass factors for each region to calculate lightning nitrogen dioxide (LNO$_{\ch{2}}$) column densities.
Our results demonstrate that LNO$_{\ch{2}}$ column densities in the Arctic are comparable to NO$_{\ch{2}}$ column densities in heavily polluted regions for the span of a few hours.
We find that the LNO$_{\ch{2}}$ production efficiency over the Arctic ocean is $\sim$5 times higher than over Arctic land, and summer LNO$_{\ch{x}}$ emissions are 5\% of anthropogenic NO$_{\ch{x}}$ emissions in the Arctic.
Additionally, we observe similar LNO$_{\ch{2}}$ production efficiencies in both the Arctic and polluted regions.
By analyzing high NO$_{\ch{2}}$ data from dissipated convection systems, we also obtain insights into LNO$_{\ch{2}}$ lifetimes and production efficiencies.

Secondly, using TROPOMI observations of convective clouds at different heights in mid-low latitude regions, we reveal the ``C''-shaped NO$_{\ch{2}}$ profiles within clouds.
Comparing TROPOMI observations with results from a global chemistry model, we find that the upper tropospheric O$_{\ch{3}}$ concentration under cloudy conditions is 26\% lower in mid-latitude regions compared to clear-sky conditions, and it decreases by 17\% in low-latitude marine areas.
However, the average O$_{\ch{3}}$ concentration in central Africa and northern India increases by 20\% compared to clear-sky conditions.
Additionally, we point out that global models underestimate the production of LNO$_{\ch{2}}$ and the vertical transport of anthropogenic pollutants, resulting in a 10--50\% underestimation of upper tropospheric NO$_{\ch{2}}$.
Therefore, the profile information obtained from TROPOMI observations can be used to evaluate the results of chemical transport models and guide the development of parameterization schemes.

Finally, we investigate the changes in dynamic transport, chemical reactions, and reaction rates in convective systems in southeastern China by integrating data from TROPOMI observations, WRF-Chem simulations, and O$_{\ch{3}}$ sounding experiments.
Our findings reveal that dynamic transport can dominate O$_{\ch{3}}$ concentration changes during the convective active period,
but chemical reactions contribute 5--10 times more than dynamic transport over the entire lifetime.
% with LNO$_{\ch{x}}$ increasing the accumulated net chemical reaction rate of upper tropospheric O$_{\ch{3}}$ by 30\%.
Furthermore, if we consider LNO$_{\ch{2}}$ in the a priori profile used for TROPOMI NO$_{\ch{2}}$ inversion, we observe a decrease of tropospheric air mass factors by 23\% in the fresh lightning area, and an increase by 60\% in the outflow and aging areas.
}
