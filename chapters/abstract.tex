%!TEX root = ../thesis.tex

\abstract
%%%%%%%%%%%%%%%%%%%%%%%%%%%%%%%%%%%%%%%%%%%%%%%%%%%%%%
%%
%%  					中文摘要
%%
%%%%%%%%%%%%%%%%%%%%%%%%%%%%%%%%%%%%%%%%%%%%%%%%%%%%%%
{
深对流云是大气垂直输送的主要载体,它能将污染气体在较短时间内,
由边界层输送至对流层上层甚至平流层下层,从而影响区域和全球的大气环境与气候。
此外闪电氮氧化物(LNO$_{\ch{x}}$)是对流层上层氮氧化物(NO$_{\ch{x}}$)的主要来源,目前LNO$_{\ch{x}}$的产率及其对臭氧(O$_{\ch{3}}$)的影响仍有很大的不确定性。
由于探空和飞机观测等传统试验的难度较大,
所以卫星遥感技术的发展为定量其排放提供了新方法。
本研究围绕这一特点,通过卫星资料反演、数值模拟和臭氧探空试验,
确定了污染和清洁地区闪电二氧化氮(LNO$_{\ch{2}}$)和LNO$_{\ch{x}}$的产率,
对比分析了观测和模拟的二氧化氮(NO$_{\ch{2}}$)与O$_{\ch{3}}$垂直分布,
揭示了深对流的动力输送和化学反应对NO$_{\ch{2}}$和O$_{\ch{3}}$变化的贡献。

首先,利用对流层观测仪(TROPOMI)和臭氧监测仪(OMI)遥感观测资料,
开发了同时适用于污染和清洁地区LNO$_{\ch{2}}$和LNO$_{\ch{x}}$柱浓度的反演算法。
结合该柱浓度产品和地基闪电观测资料,
得到了北美大陆地区旺盛对流的LNO$_{\ch{2}}$和LNO$_{\ch{x}}$产率,
分别为32$\pm$15 mol NO$_{\ch{2}}$每闪电和90$\pm$50 mol NO$_{\ch{x}}$每闪电。
将该结果与前人在北美地区的星载观测研究进行对比验证,揭示了本研究中反演算法的普适性。
除普遍研究的旺盛对流之外,该反演算法也适用于对流的初始和消散阶段,
弥补了TROPOMI在旺盛对流处因饱和效应而缺失数据的缺点。
结果显示中国东南部污染地区消散对流的LNO$_{\ch{x}}$产率为60$\pm$33 mol NO$_{\ch{x}}$每闪电。
清洁的北极陆地地区LNO$_{\ch{2}}$产率为2.0 mol每闪击,与中纬度污染地区的LNO$_{\ch{2}}$产率相当。
而北极海洋地区的LNO$_{\ch{2}}$ 产率是北极陆地性闪电的6倍。
总体来看北极夏季的LNO$_{\ch{x}}$排放量相当于该地区人为NO$_{\ch{x}}$排放的5\%。

其次,利用TROPOMI针对不同高度对流云的观测和云切片算法,得到了对流层顶--330 hPa、330--450 hPa、450--570 hPa、570--670 hPa、670--770 hPa和770--870 hPa各层NO$_{\ch{2}}$和O$_{\ch{3}}$平均浓度,
揭示了中低纬度对流云内的“C”型NO$_{\ch{2}}$垂直分布廓线。
具体而言,陆地地区对流层顶--330 hPa间的NO$_{\ch{2}}$浓度为450--570hPa间的$\sim$2倍,570hPa以下的NO$_{\ch{2}}$浓度逐渐上升,
即LNO$_{\ch{2}}$在对流层上层占主导,而污染 NO$_{\ch{2}}$ 在对流下层占主导。
此外,全球模式低估了LNO$_{\ch{2}}$和人为污染物垂直输送,
导致模拟的对流层上层NO$_{\ch{2}}$浓度偏低10--50\%。
通过对比有云和晴空条件下的TROPOMI观测数据和全球模式资料,
发现有云条件下对流层上层O$_{\ch{3}}$平均浓度在中纬度地区下降了26\%,
在低纬度海洋地区下降了17\%,
在非洲中部由于生物质燃烧而升高了20\%。
因此TROPOMI观测获得的廓线信息可用于评估模式并指导参数化方案的开发。

最后,综合TROPOMI观测、WRF-Chem模拟和O$_{\ch{3}}$探空试验数据,
分析了中国东南部不同类型对流系统中动力输送、化学反应和化学反应速率的时空变化。
结果表明,虽然动力输送项在对流旺盛期间主导了O$_{\ch{3}}$浓度变化,
但化学反应在整个生命期间的贡献可达动力输送项的5--10倍,
该现象与对流前后臭氧探空观测的对比结果相符。
此外敏感性试验表明,LNO$_{\ch{x}}$造成对流层上层O$_{\ch{3}}$化学累积生成速率降低 4\%,累积消耗速率增加23\%,
进而导致O$_{\ch{3}}$平均浓度降低25\%。
若将对流分为核心区和层云区,则核心区的动力输送作用为层云区的$\sim$2倍,
层云区的O$_{\ch{3}}$变化受核心区的输送控制。
虽然LNO$_{\ch{x}}$使得核心区的O$_{\ch{3}}$化学产量增加125\%,
但净产量下降21\%。
% 若将LNO$_{\ch{2}}$考虑进TROPOMI NO$_{\ch{2}}$反演所使用的先验廓线,
% 则对流层空气质量因子在新生闪电区降低23\%,
% 而在出流区和老化区增大60\%。

以上结果揭示了LNO$_{\ch{x}}$产率的区域差异性,
分析了深对流对对流层NO$_{\ch{2}}$和O$_{\ch{3}}$浓度变化的机理与贡献。
因此在全球气候变暖的背景下,我们需要在多卫星平台和模式的基础上,
建立更为完善的评估系统,从而增进对深对流活动及其影响的认知。
}
%%%%%%%%%%%%%%%%%%%%%%%%%%%%%%%%%%%%%%%%%%%%%%%%%%%%%%
%%
%%  					英文摘要
%%
%%%%%%%%%%%%%%%%%%%%%%%%%%%%%%%%%%%%%%%%%%%%%%%%%%%%%%
{
Deep convection clouds play a crucial role in transporting mass and pollutants in the atmosphere.
They can transport the pollutants-containing air from the boundary layer to the upper troposphere or even the lower stratosphere in a relatively short time, thus affecting regional and global atmospheric environment and climate.
In addition, lightning nitrogen oxides (LNO$_{\ch{x}}$) are the major source of nitrogen oxides (NO$_{\ch{x}}$) in the upper troposphere
and affect ozone (O$_{\ch{3}}$) chemistry, yet their production and impact remain uncertain.
Conventional observation methods (e.g. sounding and aircraft observations) are challenging for convection studies,
but satellite remote sensing technology provides new opportunities for quantifying emissions.
This study uses satellite data inversion, numerical simulation, and ozonesonde experiments to determine LNO$_{\ch{2}}$ and LNO$_{\ch{x}}$ production efficiencies in clean and polluted regions,
and analyze the observed and simulated nitrogen dioxide (NO$_{\ch{2}}$) and O$_{\ch{3}}$ vertical distributions.
It sheds light on the relative impact of transportation and chemical reactions on NO$_{\ch{2}}$ and O$_{\ch{3}}$ concentration changes.

Firstly, we develop an inversion algorithm of LNO$_{\ch{2}}$ and LNO$_{\ch{x}}$ column densities for polluted and clean areas, using the Tropospheric Monitoring Instrument (TROPOMI) and the Ozone Monitoring Instrument (OMI) observations.
Combining this product with ground-based lightning observations, the LNO$_{\ch{2}}$ and LNO$_{\ch{x}}$ production efficiencies in the continental US are determined to be 32$\pm$15 mol NO$_{\ch{2}}$ per flash and 90$\pm$50 mol NO$_{\ch{x}}$ per flash, respectively.
Comparing these results with previous space-based studies validate the universality of the inversion algorithm used in this study.
In addition to active convection, this algorithm is also applicable to the initial and dissipated convection, overcoming the limitation of TROPOMI pixel saturation issue over active convection.
Results show that the LNO$_{\ch{x}}$ production efficiency for dissipated convection is 60$\pm$33 mol NO$_{\ch{x}}$ per flash in Southeast China.
Additionally, we observe similar LNO$_{\ch{2}}$ production efficiencies in the Arctic (2.0 mol NO$_{\ch{2}}$ per stroke) and mid-latitude polluted regions.
The LNO$_{\ch{2}}$ production efficiency over the Arctic ocean is $\sim$5 times higher than over Arctic land, and summer LNO$_{\ch{x}}$ emissions are 5\% of anthropogenic NO$_{\ch{x}}$ emissions in the Arctic.

Secondly, we obtain average NO$_{\ch{2}}$ and O$_{\ch{3}}$ concentrations for the six layers (tropopause--330 hPa, 330--450 hPa, 450--570 hPa, 570--670 hPa, 670--770 hPa, and 770--870 hPa) using TROPOMI observations of convective clouds at different heights in mid-low latitude regions, and reveal the ``C''-shaped NO$_{\ch{2}}$ profiles within clouds.
Specifically, the NO$_{\ch{2}}$ concentration between the tropopause and 330 hPa over land is $\sim$ twice that of 450--570 hPa, and the NO$_{\ch{2}}$ concentration below 570 hPa gradually increases, indicating that LNO$_{\ch{2}}$ dominates in the upper troposphere, while pollutant NO$_{\ch{2}}$ dominates in the lower troposphere.
Additionally, we point out that global models underestimate the production of LNO$_{\ch{2}}$ and the vertical transport of anthropogenic pollutants, resulting in a 10--50\% underestimation of upper tropospheric NO$_{\ch{2}}$.
Comparing TROPOMI observations with results from a global chemistry model, we find that the upper tropospheric O$_{\ch{3}}$ concentration under cloudy conditions is 26\% lower in mid-latitude regions compared to clear-sky conditions, and it decreases by 17\% in low-latitude marine areas.
However, the average O$_{\ch{3}}$ concentration in central Africa increases by 20\% compared to clear-sky conditions.
Therefore, the profile information obtained from TROPOMI observations can be used to evaluate the results of chemical transport models and guide the development of parameterization schemes.

Finally, we investigate the changes in dynamic transport, chemical reactions, and reaction rates in convective systems in southeastern China by integrating data from TROPOMI observations, WRF-Chem simulations, and ozonesonde experiments.
Our findings reveal that dynamic transport dominates O$_{\ch{3}}$ concentration changes during the active period,
but chemical reactions contribute 5--10 times more than dynamic transport over the entire lifetime.
This phenomenon is consistent with the ozonesonde observations before and after convection.
In addition, sensitivity experiments show that LNO$_{\ch{x}}$ reduce the chemical production rate of upper tropospheric O$_{\ch{3}}$ by 4\% and increase the integrated production rate by 23\%,
leading to a 25\% decrease in the average O$_{\ch{3}}$ concentration.
If the convection system is divided into convective and stratiform regions,
the dynamic transport in the convective area is $\sim$ twice that in the stratiform area,
and the O$_{\ch{3}}$ changes in the stratiform area are controlled by the transport from the convective core.
Although LNO$_{\ch{x}}$ increases the chemical production of O$_{\ch{3}}$ in the convective area by 125\%, the net production decreases by 21\%.

Given the context of global warming,
it is crucial that we need to establish a more comprehensive evaluation system based on multiple satellite platforms and models,
in order to enhance our understanding of deep convection activity and its impacts.
}
