%!TEX root = ../thesis.tex

\abstract
%%%%%%%%%%%%%%%%%%%%%%%%%%%%%%%%%%%%%%%%%%%%%%%%%%%%%%
%%
%%  					中文摘要
%%
%%%%%%%%%%%%%%%%%%%%%%%%%%%%%%%%%%%%%%%%%%%%%%%%%%%%%%
{
深对流云是大气质量垂直输送的主要载体,它能够使各种污染气体在相对较短的时间内,
由边界层源区输送到对流层上层甚至平流层低层,从而影响区域和全球大气环境和气候。
此外闪电氮氧化物(LNO$_x$)是对流层上层氮氧化物(NO$_x$)的主要来源,对臭氧(O$_3$)化学产生直接和间接的影响。
目前LNO$_x$的产率、分布及其影响仍具有很大的不确定性,
基于探空和飞机观测等传统方法的对流观测试验难度较大,而卫星遥感技术的发展为定量排放提供了新方法。
本研究围绕这一特点,通过卫星资料分析、数值模拟和外场观测试验,
确定了不同污染地区的LNO$_x$产率和产量,
对比分析了观测和模拟的二氧化氮(NO$_2$)及O$_3$垂直分布,
揭示了动力输送和化学反应对对流层上层NO$_2$和O$_3$浓度的相对重要性。

首先,基于对流层观测仪(TROPOMI)和臭氧监测仪(OMI)的NO$_2$观测,
针对清洁地区(北极)和污染地区(北美大陆和中国东南部)的旺盛对流,
分别定义了各自用于计算闪电二氧化氮(LNO$_2$)柱浓度的大气质量因子,
结果显示在小时尺度上LNO$_2$柱浓度与重污染地区的NO$_2$柱浓度相当,
北极海洋上的LNO$_2$产率是北极陆地性闪电的6倍,北极夏季的LNO$_x$排放量是该地区人为NO$_x$排放的5\%。
此外结果分析表明北极地区与污染地区的LNO$_2$产率相当。
除旺盛对流之外,消散期的对流系统仍存在NO$_2$高值区,可进一步得到北极地区的LNO$_2$寿命和污染地区的LNO$_2$产率。

其次,利用TROPOMI针对不同高度对流云的NO$_2$和O$_3$观测,
获得了中低纬度地区对流云内NO$_2$和O$_3$的垂直分布。
通过对比TROPOMI观测数据和全球化学模式结果,验证了LNO$_2$在对流层上层的主导地位,
揭示了全球模式对于LNO$_2$和污染物垂直输送模拟的不足,
指出了该廓线信息可用于评估化学传输模型的结果,并指导参数化方案的开发。

最后,综合TROPOMI观测、WRF-Chem模拟和O$_3$探空试验,
定量分析了中国东南部典型对流系统中动力输送、化学反应和化学反应速率的时空变化。
结果表明,虽然在对流旺盛期间动力输送项主导O$_3$的浓度变化,但在整个生命期中化学反应更为重要。
其中LNO$_x$使得对流层上层NO$_2$浓度和累积净化学反应速率增大,但动力输送和化学反应的综合作用依然降低了对流层上层O$_3$浓度。
}
%%%%%%%%%%%%%%%%%%%%%%%%%%%%%%%%%%%%%%%%%%%%%%%%%%%%%%
%%
%%  					英文摘要
%%
%%%%%%%%%%%%%%%%%%%%%%%%%%%%%%%%%%%%%%%%%%%%%%%%%%%%%%
{
Deep convection clouds play a crucial role in transporting mass and pollutants in the atmosphere, impacting regional and global environment.
Lightning nitrogen oxides (LNO$_x$) are the main source of nitrogen oxides (NO$_x$) in the upper troposphere
and affect ozone (O$_3$) chemistry, yet their production, distribution, and impact remain uncertain.
Conventional observation methods (e.g. sounding and aircraft observations) are challenging for convection studies,
but satellite remote sensing provides new opportunities for quantifying emissions.
This study uses satellite data analysis, numerical simulation, and field experiments to determine LNO$_x$ production efficiencies and emissions in clean and polluted regions,
and analyze the observed and simulated nitrogen dioxide (NO$_2$) and O$_3$ vertical distributions.
It sheds light on the relative impact of dynamic transportation and chemical reactions on NO$_2$ and O$_3$ concentration in the upper troposphere.

Firstly, we analyze NO$_2$ observations from the TROPOspheric Monitoring Instrument (TROPOMI) and
Ozone Monitoring Instrument (OMI) instruments in both clean regions, such as the Arctic, and polluted areas such as North America and southeastern China, during periods of active convection.
To calculate the lightning nitrogen dioxide (LNO$_2$) column densities, air mass factors are determined for each region.
Results show that LNO$_2$ column densities in the Arctic are comparable to NO$_2$ column densities in heavily polluted regions for the span of a few hours.
Our findings reveal that the LNO$_2$ production efficiency over the Arctic ocean is $\sim$5 times higher than over Arctic land, and summer LNO$_x$ emissions are 5\% of anthropogenic NO$_x$ emissions in the Arctic.
Additionally, we find similar LNO$_2$ production efficiencies in both the Arctic and polluted regions.
By analyzing high NO$_2$ data from dissipated convection systems, we can gain insights into LNO$_2$ lifetimes in the Arctic and production efficiencies in polluted regions.

Secondly, we obtain NO$_2$ and O$_3$ profiles within clouds using TROPOMI observations of convection clouds at different heights in the mid-low latitude regions.
Our results demonstrate that LNO$_2$ is prevalent in the upper troposphere, as confirmed by comparisons with global chemical simulation results.
However, these comparisons also reveal limitations in the simulation of LNO$_2$ and the vertical transport of pollutants in the global model.
Thus, this profile information can be used to assess chemical transport models and guide the development of parameterization schemes.

Finally, by combining data from TROPOMI observations, WRF-Chem simulations, and O$_3$ sounding experiments,
we examine the temporal and spatial changes in dynamic transport, chemical reactions, and reaction rates in convective systems in southeastern China.
Our results indicate that while dynamic transport plays a significant role in altering O$_3$ concentrations during active periods, chemical reactions have a more profound effect on the entire life cycle of convection.
LNO$_x$ elevates NO$_2$ concentrations in the upper troposphere and the net chemical reaction rate,
but the combined influence of dynamic transport and chemical reactions still results in a decrease in upper-troposphere O$_3$ concentrations.
}
