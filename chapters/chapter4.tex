%!TEX root = ../thesis.tex

\chapter{闪电氮氧化物产率的估算} \label{chapter:PE}

\section{清洁地区(北极)}

\subsection{闪电的分布}
\subsection{反演闪电氮氧化物的步骤}
\subsection{闪电氮氧化物的聚类}
\subsection{闪电氮氧化物浓度与其他氮氧化物源的对比}
\subsection{闪电氮氧化物的寿命及产率}
\subsection{不确定性分析}


\section{污染地区(中国及美国大陆)}

\subsection{模式设置}

使用的WRF-Chem版本为3.5.1,水平网格大小为12 km $\times$ 12 km (图 ??(a)),垂直层数为29层,时间步长为72 s。
气象条件的初始场和边界场为时间分辨率3小时的北美区域再分析(NARR)数据集,每3小时应用一次边界条件和四维数据分析(FDDA)逼近,
其中温度、水汽和水平风以0.0003 s$^{-1}$ 的系数逼近\citep{Laughner.2017}。
微物理过程采用Lin方案\citep{Lin.1983},积云参数化为Grell 3D方案\citep{Grell.1993a,Grell.2002a},长波辐射采用RRTM方案\citep{Iacono.2008},短波辐射采用Goddard方案,陆面过程使用Noah陆面模式\citep{Koren.1999},边界层采用YSU方案\citep{Hong.2006}。
闪电参数化采用基于对流参数化的中性浮力水平\citep{Pickering.1992},云闪与地闪的比例基于\citet{Boccippio.2001}.

采用臭氧和相关化学示踪剂模型第4版(MOZART-4;\citet{Emmons.2010})的输出场作为化学的初始场和边界场。
人为排放由2011年美国国家排放清单(NEI)驱动,并根据环境保护署年度总排放量,按模拟的年份进行调整\citep{EPA.2015}。
生物排放使用MEGAN源,化学机制是区域大气化学机制第2版(RACM2;\citet{Goliff.2013}),并由\citet{Browne.2014}和\citet{Schwantes.2015}进行了更新。
此外,LNO$_x$参数化采用每次闪电产生200 mol NO,调整因子为1,以下简称“1$\times$200 mol NO per flash”)。
基于\citet{Ott.2010}的双峰型闪电NO(LNO)廓线\citep{Laughner.2017}被用作 WRF-Chem中LNO的垂直分布,而LNO和LNO$_2$廓线是指有和没有闪电的模拟之间垂直廓线的差异。


\subsection{反演闪电氮氧化物的步骤}

首先我们使用恒定值网格化法,将公式(\ref{eq:AMF_LNO2})所得的LNO$_x$垂直柱密度(V$_{\textrm{NO$_x$}}$)分配至0.05$^{\circ}$ $\times$ 0.05$^{\circ}$网格\citep{Kuhlmann.2014}。
接着在 1$^{\circ}$ $\times$ 1$^{\circ}$的网格中进行分析,要求每个网格至少有50个有效的0.05$^{\circ}$ $\times$ 0.05$^{\circ}$网格数据,从而最小化噪点数据。
具体LNO$_x$的主要计算步骤如下。

云辐射分数(CRF,CRF $\geq$ 70\%,CRF $\geq$ 90\%,CRF = 100\%)和 云压(CP,CP $\leq$ 650 hPa)是OMI像素是否包含深对流云的判断标准\citep{Ziemke.2009,Choi.2014,Pickering.2016}。
不同CRF对LNO$_x$产品的影响将在\ref{subsec:criteria}节探讨。
此外,我们将另一个云分数 (CF) 标准应用于 WRF-Chem 的模拟结果,以确保对流被成功模拟。
具体而言,CF是由 Xu-Randall 方法计算的 350--400 hPa 之间的最大云分数\citep{Xu.1996,Strode.2017}。
选择350--400 hPa的大气层,可避免模拟高云中的偏差。
我们选择\citet{Strode.2017}建议的 CF $\geq$ 40\%来判断模拟所处的网格是多云或晴空。

除了云特性之外,OMI能探测到新生LNO$_x$的另一条件为一段时间内有足够的闪电(或闪击)。
其中,时间窗口 (t$_{window}$) 是 OMI 过境之前的时间段。
\citet{Lapierre.2020}利用1$^{\circ}$ $\times$ 1$^{\circ}$网格的对角线长度和OMI过境时美国大陆上空500--100 hPa的平均风速计算得到t$_{window}$为2.4 h。
同时,\citet{Lapierre.2020}定义在t$_{window}$时间段内,发生2400次闪电或8160次闪击的网格才能提供足够的LNO$_x$给OMI探测。
\citet{Bucsela.2019}研究表明,低频率的闪电具有更高的LNO$_x$产率(PE),而该段数据被\citet{Lapierre.2020}使用的标准所剔除。
由于我们的研究重点是开发一种新的空气质量转换因(AMF),并将结果与使用类似闪电阈值的其他产品进行比较(Lapierre 等人,2020;Pickering 等人,2016),
因此我们将仅根据正文中的严格标准讨论结果。
为了比较每箱 2400 次闪光和每箱 1 次闪光的标准,附录 B 中提供了使用不同闪电标准的散点图。


\subsection{适合反演闪电氮氧化物的条件} \label{subsec:criteria}



\subsection{不同反演方法的对比}

\subsection{背景氮氧化物浓度对反演的影响}

\subsection{不确定性分析}

\section{本章小结}
