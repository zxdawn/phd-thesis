%!TEX root = ../thesis.tex

\chapter{闪电氮氧化物产率的估算}

\section{清洁地区(北极)}

\subsection{闪电的分布}
\subsection{闪电氮氧化物的聚类}
\subsection{闪电氮氧化物浓度与其他氮氧化物源的对比}
\subsection{闪电氮氧化物的寿命及产率}
\subsection{不确定性分析}


\section{污染地区(中国及美国大陆)}

\subsection{模式设置}

使用的WRF-Chem版本为3.5.1,水平网格大小为12 km $\times$ 12 km (图 ??a),垂直层数为29层,时间步长为72 s。
气象条件的初始场和边界场为时间分辨率3小时的北美区域再分析(NARR)数据集,每3小时应用一次边界条件和四维数据分析(FDDA)逼近,
其中温度、水汽和水平风以0.0003 s$^{-1}$ 的系数逼近\citep{Laughner.2017}。
微物理过程采用Lin方案\citep{Lin.1983},积云参数化为Grell 3D方案\citep{Grell.1993a,Grell.2002a},长波辐射采用RRTM方案\citep{Iacono.2008},短波辐射采用Goddard方案,陆面过程使用Noah陆面模式\citep{Koren.1999},边界层采用YSU方案\citep{Hong.2006}。
闪电参数化采用基于对流参数化的中性浮力水平\citep{Pickering.1992},云闪与地闪的比例基于\citet{Boccippio.2001}.

采用臭氧和相关化学示踪剂模型第4版(MOZART-4;\citet{Emmons.2010})的输出场作为化学的初始场和边界场。
人为排放由2011年美国国家排放清单(NEI)驱动,并根据环境保护署年度总排放量,按模拟的年份进行调整\citep{EPA.2015}。
生物排放使用MEGAN源,化学机制是区域大气化学机制第2版(RACM2;\citet{Goliff.2013}),并由\citet{Browne.2014}和\citet{Schwantes.2015}进行了更新。
此外,LNO$_x$参数化采用每次闪电产生200 mol NO,调整因子为1,以下简称“1$\times$200 mol NO per flash”)。
基于\citet{Ott.2010}的双峰型闪电NO(LNO)廓线\citep{Laughner.2017}被用作 WRF-Chem中LNO的垂直分布,而LNO和LNO$_2$廓线是指有和没有闪电的模拟之间垂直廓线的差异。


\subsection{适合反演闪电氮氧化物的条件}
\subsection{不同反演方法的对比}
\subsection{背景氮氧化物浓度对反演的影响}
\subsection{不确定性分析}

\section{本章小结}
