%!TEX root = ../thesis.tex

\chapter{绪论}

\section{研究背景及意义}

大气污染物的环境、气候效应不仅取决于其浓度的多少,在很大程度上还决定于其在大气中的垂直分布。而深对流是将边界层空气及气溶胶传输到对流层上层和平流层下层(UTLS)的重要机制\citep{Chatfield.1984,Dickerson.1987,Pickering.1989,Yin.2002}。
在深对流系统中,污染物从地面传输到对流层上层只需要几分钟至一个小时\citep{Skamarock.2000}。
研究学者通过结合外场观测\citep{Dickerson.1987,Pickering.1996,Bertram.2007,Apel.2012,Pan.2017}和卫星资料\citep{Halland.2009,Barth.2012,Livesey.2013,Jensen.2015},现已证明深对流输送会影响UTLS的水汽和化学成分。

通过对全球对流层大气成分收支情况的估计, \citet{Cotton.1995}指出,由对流造成的边界层大气垂直输送每年大约有90 次,并且其中很大一部分能够到达对流层上层。
在低纬度地区,对流以大约每天8\%的速率向对流层上层输送,这与由光化学过程控制的氢氧化物(HO$_x$,羟基(HO)和过氧化氢自由基(HO$_2$)的总和)和氮氧化物(NO$_x$,一氧化氮(NO)和二氧化氮(NO$_2$)的总和)的产生率相当,深对流甚至可把边界层大气直接输送到对流层顶或平流层低层\citep{Prather.1997}。
其中,臭氧(O$_3$)前体物的垂直传输显著提高了上对流层云砧出流区中O$_3$的产率\citep{Pickering.1990,Pickering.1992,Pickering.1992a},由于在上对流层风速更大,且污染气体具有比在大气低层更长的生命期或滞留时间,所以前体物的影响范围比在边界层内更广。
就气候辐射强迫而言,对流层O$_3$是第三重要的温室气体,仅次于二氧化碳(CO$_2$)和甲烷(CH$_4$)\citep{Myhre.2013}。

与此同时,从边界层侵入平流层的湿空气,会增加平流层下层水汽\citep{Homeyer.2014},这是平流层下层水汽变化的主要原因之一。
而平流层水汽是年代纪全球气候变化的关键驱动力\citep{Solomon.2010}。 此外,深层对流传输也会影响气溶胶垂直分布,这是气溶胶辐射强迫的重要组成部分\citep{Mishra.2012,Park.2015}。
综上所述,对流输送可改变高层污染气体和气溶胶在区域和全球大气中的循环\citep{Clarisse.2011},并通过云和降水过程改变边界层气溶胶浓度来调节全球气候\citep{Taylor.1997}。

如Zel'dovich机理所述\citep{Zeldovich.1967},闪电产生的NO$_x$(LNO$_x$)在对流层上层通过热闪电通道中的氧分子(O$_2$)和氮分子(N$_2$)离解产生。
自从\citet{Liebig.1827}提出闪电固氮的机制以来,闪电一直被认为是NO$_x$的一个主要来源\citep{Hutchinson.1954}。
LNO$_x$全球产量大约为2–8 Tg N/a,是上对流层NO$_x$的主要来源\citep{Galloway.2004,Schumann.2007}。
新的化学机制表明,NO$_x$在上对流层的寿命取决于与雷暴的距离,在雷暴附近约3小时,而远离雷暴处约0.5–1.5天\citep{Nault.2016,Nault.2017},从而导致对流云出流区的O$_3$浓度增高\citep{Pickering.1996,Hauglustaine.2001,DeCaria.2005}。
因为O$_3$是强氧化剂和紫外线吸收剂\citep{Myhre.2013},所以LNO$_x$对O$_3$的贡献也会影响气候强迫。
此外,电晕放电亦可直接产生臭氧 \citep{Minschwaner.2008,Kotsakis.2017}。

因此,大气污染物在深对流云中的传输和清除过程,以及LNO$_x$的产量,对于理解大气污染物的远距离输送、与云和辐射的相互作用、以及对空气质量和气候变化的影响具有重要意义。



\section{国内外研究进展}

\subsection{深对流对痕量气体的垂直输送}

深对流云对大气化学成分的垂直输送,是近三十年来国际大气科学界一直高度关注的科学问题。
除了在TOGA-COARE\citep{Webster.1992}、TRACE-A\citep{Fishman.1996}和TRACE-P\citep{Jacob.2003}等一些重大国际合作项目中有关深对流的课题占了很大比重外,先后组织了一系列专门研究深对流云在大气化学成分再分布方面的大型综合观测试验,
其中包括于1998 年在欧洲进行的闪电与氮氧化物试验(EULINO$_x$,\citeauthor{Holler.2000} \citeyear{Holler.2000})、
2002 年在佛罗里达进行的CRYSTAL-FACE 试验\citep{Toon.2003}、
2003至2005年在巴西实施的欧共体项目TROCCINO$_x$\citep{Huntrieser.2008},
以及2012年在美国进行的深对流云和化学试验(DC3,\citeauthor{Barth.2019} \citeyear{Barth.2019})等。
在这些试验中,深对流云对气溶胶及其主要前始气体以及闪电对对流层上层NO$_x$和臭氧浓度的影响是研究的重点。
与此同时,各种尺度和复杂程度的数值模式也用来模拟和解释外场观测结果。

深对流输送边界层痕量气体至UTLS的能力,取决于各种气象和化学要素。
在1985年PRESTORM的风暴尺度观测项目中,由于深对流输送,6月13日和6月15日的风暴中,上对流层一氧化碳(CO)浓度显著增加\citep{Dickerson.1987,Pickering.1989}。
然而,在6月17日的个例中,上对流层出流区的CO混合比与背景水平相似。这是由于冷锋过境并阻止了边界层空气直接进入云层,从而边界层上方的空气主导了入流区\citep{Pickering.1988}。
因此,天气尺度背景在深对流输送作用中起着重要作用。

除天气尺度背景外,边界层条件和动力结构也影响着深对流的输送效率。
1989年北达科他州雷暴观测项目中的中尺度对流复合体(MCC)的模拟结果表明,边界层越潮湿,对流输送能力越强,从而更多的CO被输送至云砧\citep{Stenchikov.1996}。
其他一些对流个例研究表明,深对流的输送能力与对流的垂直速度以及传播速度密切相关\citep{Pickering.1992a,Wang.1996}。
\citet{Bigelbach.2014}通过模拟2007年美国南部大平原对流季节的传输,发现准绝热强对流(QISC)比中尺度对流系统(MCSs)具有更强和更深的通量,即不同类型的对流系统具有不同的输送能力。
\citet{Li.2017b}通过对比热泡对流,超级单体和MCS三种不同类型的对流系统,发现超级单体的痕量气体输送能力最强。
这是由于超级单体受质量通量的垂直梯度影响更大,而不是痕量气体混合比的垂直梯度。

入流区的结构也影响着深对流的输送能力。针对美国国家航空航天局(NASA)亚马逊边界层实验(ABLE)的观测,
\citet{Scala.1990}使用二维云模型研究了热带飑线内空气的运输途径。
气团轨迹分析表明,输送到云砧的空气中有50\%以上来自对流层中部(6 km或以上),而不是边界层。
进入对流核上升气流的空气中,50\%以上的气团被限制于5 km以下,并参与了4.5 km的转子循环,仅约15\%的边界层空气直接传输至12 km附近的云层。
另一方面,在亚马逊干旱季节,巴西生物质燃烧区的对流个例显示,大量的O$_3$前体物被垂直输送至上对流层,导致对流层上层O$_3$的产生大量增加\citep{Pickering.1992,Pickering.1992a,Pickering.1996}。
湿季和干季等效位温垂直廓线的巨大差异,导致对流输送特性的差异。针对中纬度地区的个例研究表明,大部分输送至UTLS的气团来源于边界层\citep{Mullendore.2005,Skamarock.2000}。

在深对流输送研究中,云参数化和云分辨模型均具有广泛应用。深对流对痕量气体的传输模拟需要真实地再现天气尺度背景、边界层结构、对流演变过程、入流区结构以及周围的化学成分。 气象研究与预报(WRF)模型是为气象研究和数值天气预报设计的三维可压非静力大气模型。气象与化学耦合模式(WRF-Chem)是结合了大气化学的WRF模型,可以同时模拟气象学中痕量气体和气溶胶的排放、传输、混合和化学反应\citep{Fast.2006,Grell.2005}。
\citet{Barth.2012}首次在整个美国大陆上以高分辨率(4 km)应用WRF-Chem模型,研究北美季风早期阶段的对流输运和化学反应。
之后,许多研究利用WRF和WRF-Chem模拟从边界层到对流云砧区的输送个例\citep{Bela.2016,Li.2017b,Li.2018}。
痕量气体的次网格对流传输是云参数化模拟的重要组成部分。
\citet{Wang.1996}针对TRACE-A观测的热带MCS和PRESTORM观测期间的飑线,利用多层嵌套探讨了次网格尺度和网格尺度对流传输的差异。
其中MCS个例使用90和30 km两层嵌套,飑线个例使用75和25 km两层嵌套。
研究发现,在上升气流中发生了大量的次网格传输(在MCS个例中约占总向上传输的41\%,在飑线个例中约占64\%)。
\citet{Ott.2009}将云分辨模式(CRM)和柱模式(SCM)对三种对流系统中痕量气体垂直分布的模拟结果进行了比较。
他们发现,相对于CRM,SCM低估了对流层上部的对流质量通量和痕量气体混合比。
此外,他们也研究了SCM中对流传输对微物理方案参数值的敏感性。
通过调整影响对流传输的最重要参数,改进了SCM对痕量气体混合比的模拟。
\citet{Freitas.2000}针对低分辨率大气模式,提出了与深对流系统相关的次网格痕量气体对流输送的参数化。
\citet{Grell.2014}详细描述了次网格对流参数化,示踪气体传输和湿清除计算方法,该方法可用于高分辨率非静力中尺度模式。
\citet{Li.2018}通过对比不同的对流参数化(12和36 km)和云分辨模拟结果,
指出对流参数化模拟的对流输送比云分辨模拟结果要弱,并且该输送能力在对流早期而不是后期受对流参数化的控制更多。
因此,对流输送参数化与次网格对流云方案需一致,且对流参数化需要能够真实反映冰相的微物理。

在上对流层中形成的O$_3$和气溶胶的多少,取决于净对流输送的液相和/或冰相中可溶并具有反应性的气体。
在上对流层中,O$_3$的形成需要NO$_x$和HO$_x$,其机理为HO$_2$和有机过氧自由基(RO$_2$)将NO氧化,然后NO$_2$光解,激发态O原子与O$_2$分子结合。
但是,由于HO$_x$的寿命短,因此上对流层中HO$_x$的量取决于寿命更长的HO$_x$前体物的含量,
例如过氧化氢(H$_2$O$_2$),甲基过氧化氢(CH$_3$OOH)和甲醛(CH$_2$O)\citep{Chatfield.1984,Prather.1997},这些前体物是可溶的,且具有液相的化学源和汇\citep{Barth.2007,Carlton.2007}。
H$_2$O$_2$是由HO$_2$自由基与其自身反应形成的。 CH$_2$O和CH$_3$OOH来自于甲烷(CH$_4$)和其他碳氢化合物的氧化。
上对流层的NO$_x$除了来源于LNO$_x$,也受对流输送的边界层NO$_x$以及硝酸(HNO$_3$)影响\citep{Grassian.2005}。
而HNO$_3$很容易被云水和冰粒清除\citep{Neu.2012}。
通过比较深对流云和化学(DC3)项目中流入区和出流区内的痕量气体,可知CH$_3$OOH的清除效率(SE,12--84\%)很大程度上取决于冰相保留系数,
但H$_2$O$_2$(80--90\%)和CH$_2$O(40--60\%)并非如此\citep{Barth.2016,Bela.2016,Fried.2016}。
\citet{Cuchiara.2020}指出,与SEAC4RS观测中CH$_3$OOH清除率(4\%--27\%)相比,DC3观测中其清除率更高,该现象很大程度上可由冰相保留系数解释。
有关DC3项目的研究总结(如垂直输送,湿清除,LNO$_x$,平流层-对流层交换等)详见\citet{Barth.2019}。

我国的研究人员在深对流云对污染物的垂直输送方面也作了有意义的探索。
\citet{GaoHuiWang.1998}利用一个欧拉型硫沉降模式,研究了积云对硫污染物垂直输送的作用。
他们的结果表明,积云引起的垂直气流可使对流层高层的硫污染物浓度增加50--400\%。
\cite{LiBing.1999,LiBing.2001}则利用一个耦合的冰雹云-化学模式模拟了
我国陕西一次单体积云对流的发展过程及其对对流层O$_3$和NO$_x$等化学成分再分布的作用。他们的模拟结果表明,
云内强烈的垂直输送能在30分钟左右把低层低体积分数的O$_3$和高体积分数的NO$_2$快速、
有效地输送到对流层的上部,造成化学物种的再分布。
\citet{HuJiaYing.2019}利用带有详细分档微物理方案及液相化学的云模式,
对2014年7月30日发生在安徽滁州境内一次深对流过程进行了模拟研究。
研究结果表明,在积云发展阶段,强上升气流使得云内源层示踪气体有效地向上输送,
对流层中部强的夹卷过程及水平入流使得云外气体入云输送至主要对流区,并在垂直气流的作用下进一步影响各层示踪气体的分布。
各层示踪气体均可向上输送至对流层上部,其中对流层中部示踪气体(2.1--4.5 km、4.5--7.5 km 和 7.5--10.8 km)的向上输送作用与近地层示踪气体(0--2.1 km)的贡献相当。

总体而言,我国在大气污染物垂直输送及相关物理化学过程和机制方面的研究还很少,即使有,也比较零散,特别是缺少对对流输送相关事实的观测认知,这与近几年开展较多的地面大气污染观测和模拟研究不相称。所以,进行对大气污染物垂直输送过程、机理及其影响的研究,对全面理解和预测其环境、气候效应都具有不可替代的作用。


\subsection{闪电氮氧化物的观测和估算}

\subsection{闪电氮氧化物的影响}

\section{存在问题及本研究目标和研究内容}

