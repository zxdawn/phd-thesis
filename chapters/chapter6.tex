%!TEX root = ../thesis.tex

\chapter{结论与展望}

\section{主要结论}

深对流系统可通过强烈的上升气流和闪电氮氧化物(LNO$_x$)显著地影响大气成分,
其中闪电是上对流层中氮氧化物(NO$_x$ = NO + NO$_2$)的主要来源,
对臭氧(O$_3$)化学和羟基自由基(OH)产生直接和间接的影响。
然而目前LNO$_x$的产率(32--1100 mol NO$_x$每闪电)、分布及其影响仍具有很高的不确定性。
因此,本文将重点解决以上问题,基于OMI和TROPOMI观测,开发了适用于不同污染程度地区的LNO$_x$计算方案,
结合WRF-Chem和MERRA2-GMI模式探究了不同高度对流云中NO$_2$和O$_3$的垂直分布,
并针对中国东南部不同类型的对流系统开展了O$_3$探空实验,
进一步量化了动力输送、化学过程和LNO$_x$对O$_3$垂直分布变化的贡献。
主要结论如下:

\begin{enumerate}[label=(\arabic*), labelindent=\parindent, leftmargin=0pt, widest=0, itemindent=*, topsep=0pt, partopsep=0pt, parsep=0pt]

\item 针对清洁地区(北极)的对流系统,利用星载传感器TROPOMI连续过境的数据,引入了LNO$_2$大气质量因子的概念,
利用查找表和高斯分布计算了有云情况下的大气质量因子,并与无云部分相结合得到了LNO$_2$的垂直柱浓度,从而计算了LNO$_2$的寿命及产率。
2019--2021年6--8月的结果表明,北极地区对流附近的LNO$_2$寿命为3 h,与污染地区的LNO$_2$寿命相似。
北极陆地地区的LNO$_2$ 产率为2.0 mol每闪击,而海洋上的LNO$_2$产率是陆地性闪电的6倍。
基于该产率,计算得到了北极地区(70$^{\circ}$ N以北)夏季LNO$_x$ 的平均排放量为219 Mg (N),
约等于人为 NO$_x$ 排放量的 5\%。

\item 针对污染地区的对流系统,我们利用高分辨率的WRF-Chem NO$_2$模拟结果来取代简单的高斯分布假设,
重新定义了LNO$_2$大气质量因子,使其能同时适用于清洁和污染区域。
2014年5--8月的分析结果表明,美国大陆的夏季 LNO$_2$ 和 LNO$_x$平均产率为32$\pm$15 mol LNO$_2$每闪电、90$\pm$50 mol LNO$_x$每闪电、6$\pm$3 mol LNO$_2$每闪击,以及17$\pm$10 mol LNO$_x$每闪击。

\item 由于TROPOMI 的像素饱和效应可导致旺盛对流处无法反演NO$_2$信息,因此我们提出了计算消散阶段的LNO$_2$。
2019年和2020年中国东南部的对流个例分析结果表明,中国东南部的LNO$_x$产率为 60 $\pm$ 33 mol NO$_x$每闪电。
WRF-Chem的敏感性试验表明,如果TROPOMI的先验廓线中未考虑LNO$_2$,
NO$_2$柱浓度在新生闪电的区域会被低估,而在出流区和老化区相反。

\item 介绍了TROPOMI的云切片算法,并依此得到了不同高度对流云内的NO$_2$和O$_3$平均浓度。
结果表明,LNO$_2$在上对流层占主导,而污染 NO$_2$ 在下对流层占主导,
上对流层低O$_3$事件常发生于热带西太平洋,而非洲中部由于对流输送的边界层高O$_3$气团,所以上对流层O$_3$在有云条件下浓度更高。
从TROPOMI观测数据与MERRA2-GMI 和 TM5 模拟结果的对比分析中可以看出,
模式低估了中国南部、印度中部、和美国东南部的参数化对流垂直输送的能力或LNO$_2$的排放,
而闪电参数化高估了美国中部和非洲中部的闪电总量,分别导致对应高度O$_3$浓度的低估和高估。

\item 由于云切片算法在中纬度地区数据较少,针对中国东南部的对流系统开展了O$_3$探空观测,
观测数据和WRF-Chem模拟结果均表明,对流发生后上对流层的O$_3$浓度和Q$_v$均增大。
详细的趋势分析指出,虽然在对流旺盛期间动力输送项主导O$_3$的浓度变化,但在整个生命期中化学反应更为重要。
此外敏感性试验表明,LNO$_x$使得累积净化学反应速率增大,但动力输送和化学反应的综合作用降低了上对流层O$_3$浓度。

\end{enumerate}

\section{论文特色与创新}

\begin{enumerate}[label=(\arabic*), labelindent=\parindent, leftmargin=0pt, widest=0, itemindent=*, topsep=0pt, partopsep=0pt, parsep=0pt]

\item 利用TROPOMI在北极地区连续过境的特性,建立了相邻时刻LNO$_2$柱浓度的关系式,定量了北极地区LNO$_2$的寿命和产率,
并揭示了海洋和陆地性LNO$_2$产率及排放的差异。

\item 利用WRF-Chem的高分辨率模式结果,定义了LNO$_2$大气质量因子,得到了污染地区卫星观测的LNO$_2$柱浓度,
定量了污染地区LNO$_2$的产率,并指出产率与闪电频率之间的幂律关系。

\item 基于前人研究的云切片算法,得到了不同高度对流云内的NO$_2$和O$_3$浓度,分析了LNO$_2$在其中的作用,
并指出了全球模式中闪电模拟和对流垂直输送参数化的不确定性。

\item 针对中国东南部对流开展了O$_3$探空试验,定量了对流前后O$_3$垂直分布的变化,并结合WRF-Chem分析了动力输送、化学反应和LNO$_x$在O$_3$浓度变化中各自所起的作用。

\end{enumerate}



\section{不足之处与展望}

本研究工作目前还存在一些不足之处和未解决的问题,围绕这些问题,在未来的工作中对以下几个问题加以解决和完善:

\begin{enumerate}[label=(\arabic*), labelindent=\parindent, leftmargin=0pt, widest=0, itemindent=*, topsep=0pt, partopsep=0pt, parsep=0pt]

\item 虽然TROPOMI在北极地区的连续过境特性可用于建立LNO$_2$的变化关系式,但由于相隔时间($\sim$100 min)较长导致不确定因素较多,如对流面积、其他NO$_x$源的输送以及太阳高度角变化。未来可以通过静止化学监测卫星的高时空分辨率数据,以提高LNO$_2$估算结果。

\item 由于WRF-Chem一般用于区域尺度的模拟,时空精度愈高,计算量愈大,无法快速得出污染地区LNO$_2$的柱浓度。
未来可以将该方法移植至全球化学模式(如GOES-Chem、TM5和GMI等),为与TROPOMI类似的星载传感器提供更为精确的先验廓线,
从而得到可用于直接研究LNO$_2$排放和影响的产品。

\item 本研究中的云切片算法采用几何大气质量因子,该因子在对流层底层误差较大,未来可用详细的查找表建立对应的因子,从而在保证精度的基础上加快计算速度,并融合至二级产品中。

\item 探空试验成功捕捉到对流系统引起的O$_3$廓线变化,然而捕捉到的对流系统种类较少,地区单一,未来可针对不同的对流频发地区开展联合观测。
此外,由于上对流层O$_3$主要受LNO$_2$和污染传输的影响,未来需开发可同时探测O$_3$和NO$_2$浓度的探空仪,为LNO$_2$和对流输送的影响研究提供宝贵数据。

\end{enumerate}
