%!TEX root = ../thesis.tex

\chapter{结论与展望}

\section{主要结论}

深对流系统可通过强烈的上升气流和闪电氮氧化物(LNO$_{\ch{x}}$)显著地影响大气成分,
其中闪电是对流层上层氮氧化物(NO$_{\ch{x}}$ = NO + NO$_{\ch{2}}$)的主要来源,
影响臭氧(O$_{\ch{3}}$)化学和羟基自由基(OH)浓度。
然而目前LNO$_{\ch{x}}$的产率(32--1100 mol NO$_{\ch{x}}$每闪电)、分布及其对O$_{\ch{3}}$的影响仍有很高的不确定性。
因此本研究主要重点解决以上问题,基于OMI和TROPOMI观测开发了适用于污染和清洁地区闪电二氧化氮(LNO$_{\ch{2}}$)柱浓度的反演算法,
结合MERRA2-GMI资料和WRF-Chem模式探究了NO$_{\ch{2}}$和O$_{\ch{3}}$的垂直分布,
并针对中国东南部不同类型的对流系统开展了臭氧探空试验,
进一步利用WRF-Chem量化了动力输送、化学过程和LNO$_{\ch{x}}$对O$_{\ch{3}}$变化的贡献。
主要结论如下:

\begin{enumerate}[label=(\arabic*), labelindent=\parindent, nosep, leftmargin=0pt, widest=0, itemindent=*, topsep=0pt, partopsep=0pt, parsep=0pt]

\item 针对污染地区的旺盛对流系统,我们利用高分辨率的WRF-Chem NO$_{\ch{2}}$模拟结果定义了LNO$_{\ch{2}}$和LNO$_{\ch{x}}$的大气质量因子,使其能同时反演清洁和污染地区的LNO$_{\ch{2}}$柱浓度。
2014年5--8月的分析结果表明,北美大陆夏季 LNO$_{\ch{2}}$ 和 LNO$_{\ch{x}}$平均产率为
32 $\pm$ 15 mol NO$_{\ch{2}}$每闪电、90 $\pm$ 50 mol NO$_{\ch{x}}$每闪电、6 $\pm$ 3 mol NO$_{\ch{2}}$每闪击以及17 $\pm$ 10 mol NO$_{\ch{x}}$每闪击。

\item 由于TROPOMI的像素饱和效应可导致旺盛对流处无法反演NO$_{\ch{2}}$柱浓度,因此我们提出了计算消散阶段的LNO$_{\ch{2}}$。
结果显示,中国东南部污染地区的LNO$_{\ch{x}}$产率为 60 $\pm$ 33 mol NO$_{\ch{x}}$每闪电。
此外WRF-Chem的敏感性试验表明,
如果将LNO$_{\ch{2}}$考虑进TROPOMI NO$_{\ch{2}}$反演所使用的先验廓线,
则对流层大气质量因子在新生闪电区降低23\%,而在出流区和老化区增加60\%。

\item 针对清洁地区(北极)的对流系统,利用TROPOMI连续过境的数据和高斯分布的LNO$_{\ch{2}}$经验廓线,简化了大气质量因子的计算,得到了北极地区LNO$_{\ch{2}}$的寿命及产率。
2019至2021年6--8月的结果表明,北极地区对流附近的LNO$_{\ch{2}}$寿命为3 h,与污染地区的LNO$_{\ch{2}}$寿命相似。
北极陆地地区的LNO$_{\ch{2}}$ 产率为2.0 mol每闪击,与中纬度
污染地区的LNO$_{\ch{2}}$产率相当,而北极海洋上的LNO$_{\ch{2}}$产率是陆地性闪电的6倍。
基于该产率,计算得到了北极地区(70$^{\circ}$ N以北)夏季LNO$_{\ch{x}}$ 平均排放量为219吨氮,
约等于人为 NO$_{\ch{x}}$ 排放量的 5\%。

\item 将云切片算法应用于TROPOMI的观测,得到了对流条件下对流层顶--330 hPa、330--450 hPa、
450--570 hPa、570--670 hPa、670--770 hPa和770--870 hPa各层NO$_{\ch{2}}$和O$_{\ch{3}}$平均浓度。
结果表明,陆地地区对流层顶--330 hPa间的NO$_{\ch{2}}$浓度为 450--570hPa间的$\sim$2倍,570hPa以下的NO$_{\ch{2}}$浓度逐渐上升,
即云内NO$_{\ch{2}}$廓线呈“C” 型,LNO$_{\ch{2}}$在对流层上层占主导,而污染 NO$_{\ch{2}}$ 在对流层下层占主导。
通过对比有云和晴空条件下的 TROPOMI 观测数据和全球
MERRA2-GMI模式资料,发现有云条件下对流层上层O$_{\ch{3}}$平均浓度在中纬度地区下降了 26\%,在低纬
度海洋地区下降了 17\%,而在非洲中部由于生物质燃烧排放升高了20\%。
因此TROPOMI观测的廓线信息可用于评估模式并指导参数化方案的开发。
从TROPOMI观测数据与MERRA2-GMI资料和TM5模拟结果的对比分析中可以看出,
模式低估了中国南部、印度中部和美国东南部的对流垂直输送能力或LNO$_{\ch{2}}$排放量,
从而导致对流层上层NO$_{\ch{2}}$偏低10--50\%。

\item 由于云切片算法在中纬度地区数据较少,针对中国东南部的对流系统开展了对流前后的臭氧探空对比观测。
观测数据和WRF-Chem模拟结果均表明,对流发生后观测区域的对流层上层O$_{\ch{3}}$浓度和Q$_v$均增大。
详细的趋势分析指出,虽然动力输送项在对流旺盛期间主导O$_{\ch{3}}$的浓度变化,
但化学反应在整个生命期中可达动力输送的5--10倍。
敏感性试验表明,LNO$_{\ch{x}}$可使得对流层上层O$_{\ch{3}}$化学累积生成速率降低 4\%,累积消耗速率增加23\%,
导致该层O$_{\ch{3}}$的平均浓度降低了25\%。
其中对流核心区的动力输送作用为层云区的$\sim$2倍,层云区的O$_{\ch{3}}$变化受核心区的输送控制。
此外LNO$_{\ch{x}}$使得核心区的O$_{\ch{3}}$化学产量增加125\%,但净产量下降21\%。


\end{enumerate}

\section{论文特色与创新}

\begin{enumerate}[label=(\arabic*), labelindent=\parindent, nosep, leftmargin=0pt, widest=0, itemindent=*, topsep=0pt, partopsep=0pt, parsep=0pt]

\item 开发了高分辨率大气成分卫星遥感反演LNO$_{\ch{2}}$柱浓度的算法,且该算法同时适用于污染和清洁地区。

\item 开展了国内首次针对白天对流前后对流层O$_{\ch{3}}$浓度变化的探空观测试验,
并利用WRF-Chem化学模式揭示了有关化学反应对其浓度正贡献的重要性。

% \item 利用TROPOMI在北极地区连续过境的特性,建立了相邻时刻LNO$_{\ch{2}}$柱浓度的关系式,定量计算了北极地区LNO$_{\ch{2}}$的寿命和产率,
% 并揭示了海洋和陆地性LNO$_{\ch{2}}$产率及排放的差异。

% \item 利用WRF-Chem的高分辨率模式结果,定义了LNO$_{\ch{2}}$大气质量因子,得到了污染地区卫星观测的LNO$_{\ch{2}}$柱浓度,
% 定量计算了污染地区LNO$_{\ch{2}}$的产率,并指出产率与闪电频率之间的幂律关系。

% \item 基于前人研究的云切片算法,得到了不同高度对流云内的NO$_{\ch{2}}$和O$_{\ch{3}}$浓度,分析了LNO$_{\ch{2}}$在其中的作用,
% 并指出了全球模式中闪电模拟和对流垂直输送参数化的不确定性。

% \item 针对中国东南部对流开展了臭氧探空试验,定量计算了对流前后O$_{\ch{3}}$垂直分布的变化,并结合WRF-Chem分析了动力输送、化学反应和LNO$_{\ch{x}}$在O$_{\ch{3}}$浓度变化中各自所起的作用。

\end{enumerate}



\section{不足之处与展望}

本研究工作目前还存在一些不足之处和未解决的问题,围绕这些问题,在未来的工作中可对以下几点加以解决和完善:

\begin{enumerate}[label=(\arabic*), labelindent=\parindent, nosep, leftmargin=0pt, widest=0, itemindent=*, topsep=0pt, partopsep=0pt, parsep=0pt]

\item 虽然TROPOMI在北极地区的连续过境特性可用于建立LNO$_{\ch{2}}$的变化关系式,但由于相隔时间($\sim$100 min)较长,导致不确定因素较多,如对流面积、其他NO$_{\ch{x}}$源的输送以及太阳高度角变化。
未来可以利用静止化学监测卫星的高时空分辨率数据,提高全球LNO$_{\ch{2}}$估算的精度。

% \item 由于WRF-Chem一般用于区域尺度的模拟,时空精度愈高,计算量愈大,无法快速得出污染地区LNO$_{\ch{2}}$的柱浓度。
% 未来可以将该方法移植至全球化学模式(如GOES-Chem、TM5和GMI等),为与TROPOMI类似的星载传感器提供更为精确的先验廓线,
% 从而得到可用于直接研究LNO$_{\ch{2}}$排放和影响的产品。

\item 本研究中的云切片算法采用几何大气质量因子,该因子在对流层低层误差较大,未来可用详细的查找表建立对应的因子,从而在保证精度的基础上加快计算速度,并融合至二级产品中。

\item 探空试验成功观测到对流系统引起的O$_{\ch{3}}$廓线变化,然而捕捉到的对流系统种类较少,地区单一,未来可针对不同的对流频发地区开展联合观测。
此外由于对流层上层O$_{\ch{3}}$主要受LNO$_{\ch{2}}$和污染传输的影响,未来需开发可同时探测O$_{\ch{3}}$和NO$_{\ch{2}}$浓度的探空仪,为研究LNO$_{\ch{2}}$和对流输送的影响提供宝贵数据。

\end{enumerate}
