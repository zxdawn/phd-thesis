%!TEX root = ../thesis.tex

\chapter{结论与展望}

\section{主要结论}

深对流系统能够通过强烈的上升气流和闪电氮氧化物(LNO$_{\ch{x}}$)对大气成分产生显著影响。
其中,闪电是对流层上层氮氧化物(NO$_{\ch{x}}$ = NO + NO$_{\ch{2}}$)的主要来源,
进而影响臭氧(O$_{\ch{3}}$)化学和羟基自由基(OH)浓度。
但目前LNO$_{\ch{x}}$的产率(32--1100 mol NO$_{\ch{x}}$每闪电)、分布及其对O$_{\ch{3}}$的影响仍存在很高的不确定性。
因此,本研究利用OMI和TROPOMI观测资料开发了适用于污染和清洁地区闪电二氧化氮(LNO$_{\ch{2}}$)柱浓度的反演算法,
并结合MERRA2-GMI资料和WRF-Chem模式探究了NO$_{\ch{2}}$和O$_{\ch{3}}$的垂直分布。
此外,本研究还开展了针对中国东部不同类型对流系统的臭氧探空试验,
并进一步利用WRF-Chem量化了动力输送、化学过程和LNO$_{\ch{x}}$对O$_{\ch{3}}$变化的贡献。
主要结论如下:

\begin{enumerate}[label=(\arabic*), labelindent=\parindent, nosep, leftmargin=0pt, widest=0, itemindent=*, topsep=0pt, partopsep=0pt, parsep=0pt]

\item \textbf{基于OMI和TROPOMI传感器的LNO$_{\ch{x}}$反演算法}

\hspace{4ex} 鉴于目前LNO$_{\ch{x}}$的反演及应用多限定于清洁地区,
因此本研究首先针对北美大陆的对流旺盛系统,利用WRF-Chem模拟的高分辨率NO$_{\ch{2}}$结果定义了LNO$_{\ch{2}}$和LNO$_{\ch{x}}$的大气质量因子,
从而同时适用于反演清洁和污染地区的LNO$_{\ch{2}}$柱浓度。
研究分析了2014年5--8月期间的数据,结果显示北美大陆夏季 LNO$_{\ch{2}}$ 和 LNO$_{\ch{x}}$平均产率为
32 $\pm$ 15 mol NO$_{\ch{2}}$每闪电、90 $\pm$ 50 mol NO$_{\ch{x}}$每闪电、6 $\pm$ 3 mol NO$_{\ch{2}}$每闪击以及17 $\pm$ 10 mol NO$_{\ch{x}}$每闪击。

\hspace{4ex} 然而TROPOMI的像素饱和效应常引起对流旺盛处无法反演NO$_{\ch{2}}$柱浓度的问题。
本研究将开发的反演算法应用于消散阶段的观测数据,并成功计算出LNO$_{\ch{2}}$的柱浓度。
研究结果表明,中国东部污染地区的LNO$_{\ch{x}}$产率为 60 $\pm$ 33 mol NO$_{\ch{x}}$每闪电,与北美大陆地区的结果相近。
此外,WRF-Chem的敏感性试验显示,
如果将LNO$_{\ch{2}}$考虑进TROPOMI NO$_{\ch{2}}$反演所使用的先验廓线,
则对流层大气质量因子在新生闪电区降低23\%,而在出流区和老化区增加60\%。

\hspace{4ex} 由于未来五年将有不同类型的静止化学监测卫星投入使用,因此本研究选择了TROPOMI连续过境的北极清洁地区,来探究新的LNO$_{\ch{2}}$产率计算方法。
此外,北极地区近年来闪电活动有所增多,为研究该地区的LNO$_{\ch{2}}$提供了条件。
本研究利用连续过境的特性和高斯分布的LNO$_{\ch{2}}$经验廓线简化了大气质量因子的计算,%并得到了北极地区LNO$_{\ch{2}}$的寿命及产率。
% 研究结果显示,
并得到了2019至2021年三个夏季(6--8月)北极地区对流附近的LNO$_{\ch{2}}$寿命为3 h,与污染地区的LNO$_{\ch{2}}$寿命相似。
此外,北极陆地地区LNO$_{\ch{2}}$产率为2.0 mol每闪击,与污染的中纬度地区LNO$_{\ch{2}}$产率相当,
而北极海洋地区LNO$_{\ch{2}}$ 产率则是北极陆地地区的6倍。
基于以上产率,本研究计算得到北极地区(70$^{\circ}$ N以北)夏季LNO$_{\ch{x}}$ 平均排放量为219吨氮,
约等于人为 NO$_{\ch{x}}$ 排放量的 5\%。

\item \textbf{基于云切片算法的对流条件下NO$_{\ch{2}}$和O$_{\ch{3}}$的垂直分布}

\hspace{4ex} 本研究通过应用云切片算法对TROPOMI观测数据进行进一步分析,可以获得在对流事件发生时不同高度层(对流层顶至330 hPa、330至450 hPa、
450至570 hPa、570至670 hPa、670至770 hPa和770至870 hPa)的NO$_{\ch{2}}$和O$_{\ch{3}}$平均浓度。
研究结果显示,陆地地区对流层顶至330 hPa高度间的NO$_{\ch{2}}$浓度约为 450--570 hPa高度间的两倍,而在570 hPa高度以下NO$_{\ch{2}}$浓度随高度的增加而降低,
即云内NO$_{\ch{2}}$廓线呈“C” 型,在对流层上层NO$_{\ch{2}}$中LNO$_{\ch{2}}$占主导,而在对流层下层人为排放的NO$_{\ch{2}}$占主导。
从TROPOMI观测数据与MERRA2-GMI资料和TM5模拟结果的对比分析中可以看出,
MERRA2-GMI和TM5低估了中国南部、印度中部和美国东南部的对流垂直输送能力或LNO$_{\ch{2}}$排放量,
从而导致对流层上层NO$_{\ch{2}}$偏低10--50\%。
通过对比有云和晴空条件下的 TROPOMI 观测数据和全球
MERRA2-GMI模式资料,研究发现,有云时对流层上层O$_{\ch{3}}$平均浓度在中纬度地区下降了 26\%,在低纬
度海洋地区下降了 17\%,而在非洲中部,受生物质燃烧排放影响,对流层上层O$_{\ch{3}}$平均浓度升高了20\%。
因此,TROPOMI观测的廓线信息可用于模式评估并指导参数化方案的开发。

\item \textbf{中国东部地区动力输送和化学反应对O$_{\ch{3}}$垂直分布的影响}

\hspace{4ex} 由于云切片算法在中纬度地区数据较少,本研究针对中国东部的对流系统进行了对流前后的臭氧探空对比观测和模拟。
观测数据和WRF-Chem模拟结果均表明,对流发生后,观测区域的对流层上层O$_{\ch{3}}$浓度和Q$_v$均增加。
详细的趋势分析指出,虽然动力输送项在对流旺盛期间主导了O$_{\ch{3}}$浓度变化,
但化学反应在整个生命期中对O$_{\ch{3}}$的贡献可达动力输送项的5--10倍。
此外,敏感性试验表明,LNO$_{\ch{x}}$的存在使得对流层上层O$_{\ch{3}}$化学累积生成速率降低 4\%,累积消耗速率增加23\%,
导致该层O$_{\ch{3}}$的平均浓度降低了25\%。
其中,对流核心区的动力输送作用约为层云区的2倍,层云区的O$_{\ch{3}}$变化受核心区的输送控制。
此外,LNO$_{\ch{x}}$导致核心区的O$_{\ch{3}}$化学产量增加125\%,但净产量下降21\%。


\end{enumerate}

\section{论文特色与创新}

\begin{enumerate}[label=(\arabic*), labelindent=\parindent, nosep, leftmargin=0pt, widest=0, itemindent=*, topsep=0pt, partopsep=0pt, parsep=0pt]

\item 开发了具有普适性(同时适用于污染和清洁地区)的高分辨率大气成分卫星遥感反演LNO$_{\ch{2}}$柱浓度的算法。

\item 首次在国内开展了针对对流前后对流层O$_{\ch{3}}$浓度变化的探空观测试验,
并利用WRF-Chem模式探究了O$_{\ch{3}}$浓度明显升高的成因,量化了对流旺盛期间的动力输送效应,揭示了整个对流过程中化学反应对O$_{\ch{3}}$浓度正贡献的重要作用。

% \item 利用TROPOMI在北极地区连续过境的特性,建立了相邻时刻LNO$_{\ch{2}}$柱浓度的关系式,定量计算了北极地区LNO$_{\ch{2}}$的寿命和产率,
% 并揭示了海洋和陆地性LNO$_{\ch{2}}$产率及排放的差异。

% \item 利用WRF-Chem的高分辨率模式结果,定义了LNO$_{\ch{2}}$大气质量因子,得到了污染地区卫星观测的LNO$_{\ch{2}}$柱浓度,
% 定量计算了污染地区LNO$_{\ch{2}}$的产率,并指出产率与闪电频率之间的幂律关系。

% \item 基于前人研究的云切片算法,得到了不同高度对流云内的NO$_{\ch{2}}$和O$_{\ch{3}}$浓度,分析了LNO$_{\ch{2}}$在其中的作用,
% 并指出了全球模式中闪电模拟和对流垂直输送参数化的不确定性。

% \item 针对中国东部对流开展了臭氧探空试验,定量计算了对流前后O$_{\ch{3}}$垂直分布的变化,并结合WRF-Chem分析了动力输送、化学反应和LNO$_{\ch{x}}$在O$_{\ch{3}}$浓度变化中各自所起的作用。

\end{enumerate}



\section{不足之处与展望}

本研究工作目前还存在一些不足之处和未解决的问题,围绕这些问题,在未来的工作中可对以下几点加以解决和完善:

\begin{enumerate}[label=(\arabic*), labelindent=\parindent, nosep, leftmargin=0pt, widest=0, itemindent=*, topsep=0pt, partopsep=0pt, parsep=0pt]

\item 尽管TROPOMI在北极地区的连续过境特性,可用于建立LNO$_{\ch{2}}$的变化关系式,但由于相隔时间较长(约100分钟),存在许多不确定因素,例如对流面积、其他NO$_{\ch{x}}$源的输送以及太阳高度角变化。
未来,可以利用静止化学监测卫星的高时空分辨率数据来提高全球LNO$_{\ch{2}}$估算的精度。

% \item 由于WRF-Chem一般用于区域尺度的模拟,时空精度愈高,计算量愈大,无法快速得出污染地区LNO$_{\ch{2}}$的柱浓度。
% 未来可以将该方法移植至全球化学模式(如GOES-Chem、TM5和GMI等),为与TROPOMI类似的星载传感器提供更为精确的先验廓线,
% 从而得到可用于直接研究LNO$_{\ch{2}}$排放和影响的产品。

\item 本研究中的云切片算法采用几何大气质量因子,该因子在对流层低层误差较大。
未来可以使用详细的查找表建立对应的因子,从而在保证精度的基础上加快计算速度,并将其融合至二级产品中。

\item 探空试验成功观测到对流系统引起的O$_{\ch{3}}$廓线变化,但捕捉到的对流系统种类较少且地区单一。未来可针对不同的对流频发地区开展联合观测。
此外,由于对流层上层O$_{\ch{3}}$主要受LNO$_{\ch{2}}$和污染传输的影响,需要开发可同时探测O$_{\ch{3}}$和NO$_{\ch{2}}$浓度的探空仪,为研究LNO$_{\ch{2}}$和对流输送影响提供宝贵数据。

\end{enumerate}
